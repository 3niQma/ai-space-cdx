\chapter{Introduction}
\label{c:introduction}
Provide the motivation for reinforcement learning (RL) in scheduling, outline objectives and research questions, and summarize contributions and scope. Link to the systematic review protocol for transparency.
\section{Context and Motivation}
\section{Objectives and Research Questions}
\section{Contributions and Thesis Structure}

\chapter{Background}
\label{c:background}
Summarize scheduling fundamentals and RL essentials to set common ground.
\section{Scheduling Fundamentals}
Job shop, flow shop, flexible job shop, parallel machine, hybrid flow shop; deterministic vs.\ stochastic vs.\ dynamic arrivals; constraints (due dates, setup times, resource calendars).
\section{Reinforcement Learning Essentials}
MDPs, policies, value-based vs.\ policy-gradient vs.\ model-based RL; on-policy vs.\ off-policy; exploration strategies; multi-objective settings.
\section{Benchmarks and Metrics}
Makespan, tardiness, flow time, energy; OR baselines (dispatching rules, MILP/CP, metaheuristics); RL evaluation norms.
\paragraph{Planned figure} Taxonomy of scheduling problem classes.
\paragraph{Planned table} Metrics and baseline families used in RL scheduling studies.

\chapter{Methodology}
\label{c:methodology}
Describe the systematic literature review protocol and data-extraction process.
\section{Search Strategy}
Databases: IEEE Xplore, ACM Digital Library, Scopus, Web of Science, Google Scholar (for snowballing). Time window: 2016--2025. Search strings (examples, adapted per indexer): \texttt{(\"reinforcement learning\" OR \"deep reinforcement learning\" OR DQN OR PPO OR SAC) AND (scheduling OR \"job shop\" OR \"flow shop\" OR \"production scheduling\" OR \"dispatching\" OR \"production planning\" OR \"cloud scheduling\" OR \"edge scheduling\")}. Apply backward/forward snowballing on key papers and the provided surveys.
Screening follows PRISMA-style phases: deduplicate, title/abstract screen, full-text eligibility, inclusion.
\section{Inclusion and Exclusion Criteria}
\begin{itemize}
    \item \textbf{Inclusion}: peer-reviewed conference/journal papers (2016--2025) applying RL/DRL to scheduling, dispatching, production planning/control, cloud/edge scheduling; reports quantitative results against baselines.
    \item \textbf{Exclusion}: non-RL approaches, purely conceptual with no evaluation, non-English, inaccessible full text, duplicates.
\end{itemize}
\section{Quality Assessment}
Baseline strength, reproducibility (code/data), statistical validity (multiple seeds, confidence intervals), clarity of environment/problem specification, constraint handling.
\section{Data Extraction Schema}
Problem type, environment, state/action/reward, algorithm, baselines, metrics, constraints, generalization tests, code availability.
\paragraph{Planned figure} PRISMA flow diagram for study selection.
\paragraph{Planned table} Data-extraction codebook.


\chapter{Methodology (Draft Text)}
\label{c:methodology-draft}
This placeholder will be replaced by the full protocol once screening counts are known. To include: finalized search strings per database, PRISMA counts (identification/screening/eligibility/inclusion), justification for the 2016--2025 window, and the data-extraction schema aligned to the literature matrix columns. Add a PRISMA diagram and a summary table of quality assessment criteria (baseline strength, reproducibility, statistical validity, constraint handling).

\chapter{RL Methods for Scheduling: Taxonomy}
\label{c:taxonomy}
Organize the landscape of RL approaches tailored to scheduling.
\section{Value-Based Methods}
DQN/DDQN/Dueling, distributional variants; action masking for constraints.
\section{Policy-Gradient and Actor-Critic Methods}
A2C/A3C, PPO, SAC, deterministic policy gradients.
\section{Model-Based and Simulation-Augmented RL}
World models, lookahead, Dyna-style, differentiable simulators.
\section{Meta-RL, Transfer, and Curriculum Learning}
\section{State, Action, Reward Design Patterns}
Graph/state encodings, machine/job-centric actions, reward shaping for due dates/setups; constraint handling (penalties, masking, Lagrangian).
\noindent \textit{Progress note:} Initial extraction shows strong use of graph encodings (disjunctive graphs, GNN dual-attention) and action masking for feasibility in JSS/FJSS. Rewards are typically weighted makespan/tardiness, with penalties for constraint violations.
\begin{table}[t]\centering\small
  \caption{State, action, reward patterns observed in RL for scheduling}
  \label{tab:state-action-reward}
  \begin{tabular}{p{3.2cm}p{4.2cm}p{4.0cm}p{3.8cm}}
Problem class & State design & Action design & Reward design \\\hline
Job shop (JSS) static/dynamic & Disjunctive graph embeddings (GNN size-agnostic) & Dispatch next eligible operation & $-$makespan / $-$tardiness with step penalties \\
Flexible JSS (routing + sequencing) & Dual attention over operations and machines & Joint machine routing and operation sequencing & Weighted makespan + tardiness; shaping for idle time \\
Dynamic arrivals & Queue/machine status, arrival indicators & Dispatch/route arriving jobs & Weighted tardiness/completion; penalties on lateness \\
Energy-aware JSS & Machine load + energy profiles & Dispatch with energy-aware tie breaks & Combined makespan + energy cost; penalties for overconsumption \\
Cloud/edge scheduling & Resource utilization, SLA/backlog & Task-to-VM/offload assignment & $-$slowdown, latency, SLA penalties, energy terms \\
Transport/AGV & Network/vehicle positions, queue lengths & Vehicle dispatch/route choice & Throughput, delay penalties, collision avoidance penalties \\
\end{tabular}

\end{table}
\paragraph{Planned figure} Taxonomy diagram: methods vs.\ scheduling settings.
\paragraph{Planned table} State/action/reward design patterns by problem class.

\chapter{Comparative Performance Analysis}
\label{c:analysis}
Synthesize empirical results across studies, focusing on baselines, metrics, and robustness.
\section{Performance vs.\ Classical Baselines}
Dispatching rules, MILP/CP, metaheuristics; domain-wise comparison.
\begin{table}[t]\centering\small
  \caption{Baselines and metrics across domains}
  \label{tab:baselines-metrics}
  \begin{tabular}{p{3cm}p{6cm}p{4cm}}
Domain & Typical baselines & Metrics \\\hline
JSS/FJSS & PDRs (EDD, SPT, LPT), NEH, tabu/GA/SA; MILP/CP on small instances & Makespan, tardiness, total weighted tardiness (TWT) \\
Dynamic shop/fab & Dispatching rules + simulation heuristics; myopic OR heuristics & Throughput, cycle time, tardiness, service level \\
Cloud/edge & SJF, Tetris, round-robin, heuristics, OR-Tools & Makespan, slowdown, latency, SLA adherence, energy \\
Transport/AGV & Nearest-vehicle/greedy dispatch, rule-based logistics heuristics & Throughput, travel time, tardiness \\
Energy-aware & Energy-aware heuristics, metaheuristics & Energy consumption, makespan, tardiness \\
\end{tabular}

\end{table}
\section{Generalization and Robustness}
Out-of-distribution instances, dynamic arrivals, noise/perturbations.
\section{Sample Efficiency and Ablations}
Replay strategies, curriculum, reward shaping.
\noindent \textit{Progress note:} Recent GNN/PPO and dual-attention actor-critic schedulers outperform classic PDRs on JSS/FJSS benchmarks and generalize to larger unseen instances; exact OR tools still stronger on some cases.
\paragraph{Planned tables} Performance comparison per domain; robustness/generalization results; sample-efficiency summaries.
\paragraph{Planned figure} Heatmap of methods vs.\ benchmarks and win/loss vs.\ baselines.

\section{Constraint Handling and Feasibility}
Penalty shaping, masking, Lagrangian/shields. Table~\ref{tab:constraint-handling} summarizes common techniques.
\begin{table}[t]\centering\small
  \caption{Constraint-handling techniques in RL scheduling}
  \label{tab:constraint-handling}
  \begin{tabular}{p{3cm}p{5cm}p{5cm}}
Technique & Examples & Notes \\\hline
Action masking & Feasible machines/operations only; block violating routes & Stabilizes training, keeps feasibility; used in JSS/FJSS and constrained routing \\
Penalty shaping & Add cost for lateness, setups, energy overuse, SLA violation & Simple to implement; may struggle with hard constraints if penalties mis-tuned \\
Shields/filters & Safety layer vetoes unsafe actions (collisions, overruns) & Effective for safety-critical transport/production; requires rule base \\
Lagrangian/dual & Penalty multipliers updated during training & Better balance feasibility vs performance; needs tuning \\
Curricula & Start relaxed, tighten constraints over training & Improves learning stability under heavy constraints \\
\end{tabular}

\end{table}

\chapter{Application Domains}
\label{c:domains}
Short vignettes for key sectors and their specific constraints.
\section{Semiconductor and Flexible Manufacturing}
\section{Logistics and Transportation}
\section{Cloud and Edge Computing}
\section{Energy-Aware and Sustainable Scheduling}
\paragraph{Planned tables} Domain-specific datasets/benchmarks and metrics; constraint profiles per domain.
\paragraph{Planned figure} Timeline of notable RL-in-scheduling papers per domain.

\chapter{Cross-Cutting Challenges}
\label{c:challenges}
Discuss systemic issues in applying RL to scheduling.
\section{Stability and Variance}
Seed sensitivity, policy brittleness.
\section{Constraint Handling}
Hard vs.\ soft constraints, feasibility preservation, masking vs.\ penalties.
\section{Simulation-to-Real Gap}
Domain randomization, robust policies, transfer.
\section{Interpretability and Safety}
Action rationale, override strategies, safe RL.
\section{Reproducibility}
Open code/data, hyperparameter documentation, evaluation protocols.
\paragraph{Planned tables} Constraint-handling techniques; reproducibility checklist.
\paragraph{Planned figure} Sim-to-real mitigation strategies.

\chapter{Open Gaps and Future Directions}
\label{c:future}
Identify promising research avenues grounded in observed gaps.
\section{Hybrid RL and Operations Research}
Learning-augmented heuristics, RL-guided search, primal-dual methods.
\section{Offline, Safe, and Risk-Sensitive RL}
\section{Transfer, Meta-Learning, and Continual Adaptation}
\section{Benchmarking and Standardization}
Need for standardized environments, seeds, reporting.
\paragraph{Planned figure} Roadmap of future research directions and milestones.

\chapter{Conclusion}
\label{c:conclusion}
Synthesize insights, answer research questions, and highlight practical implications for deploying RL in scheduling.
