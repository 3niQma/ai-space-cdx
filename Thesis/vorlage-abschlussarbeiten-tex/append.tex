\chapter{PRISMA Documentation}
\label{app:prisma}
This appendix archives the artifacts required to reproduce the PRISMA workflow.
\begin{description}
  \item[Flow counts:] \texttt{data/prisma/flow\_counts.csv} (auto-ingested by \texttt{automation/prisma\_flow.py} to render Figure~\ref{fig:prisma-flow}).
  \item[Search log:] \texttt{data/prisma/search\_log.csv} records database, query string, filters, export date, and hit count.
  \item[Screening log:] \texttt{data/prisma/screening\_log.csv} stores \texttt{paper\_id}, decisions at abstract/full-text stages, exclusion reasons, and reviewer notes.
  \item[Automation scripts:] \texttt{automation/agent\_pipeline.py} (orchestrates agent runs), \texttt{automation/prisma\_flow.py} (chart generation), and \texttt{automation/summarize\_studies.py} (statistics).
\end{description}
All CSV files are UTF-8 encoded; schema definitions are provided in Appendix~\ref{app:templates}.

\chapter{Search Strategies}
\label{app:search}
Table~\ref{tab:search-strings} lists the search configurations executed between 2--4 Nov 2025. Queries follow database-specific syntax but share the core structure ``(\textit{reinforcement learning terms}) AND (\textit{manufacturing scheduling terms}).''

\begin{table}[t]
  \centering
  \caption{Executed database queries for the initial PRISMA cycle. For updates, extend \texttt{data/prisma/search\_log.csv}.}
  \label{tab:search-strings}
  \begin{tabular}{p{3cm} p{7cm} p{3cm} p{1.5cm}}
    \toprule
    Source & Query (abridged) & Filters & Hits \\
    \midrule
    Scopus & \texttt{TITLE-ABS-KEY ("reinforcement learning" AND "job shop scheduling")} & 2014--2024, journal+conference & 124 \\
    IEEE Xplore & \texttt{("reinforcement learning" NEAR/3 scheduling) AND ("flow shop" OR "flexible job shop")} & 2014--2024, manufacturing subject area & 68 \\
    Web of Science & \texttt{TS=("multi-agent reinforcement learning" AND scheduling AND manufacturing)} & 2014--2024, SCI-Expanded & 57 \\
    ACM DL & \texttt{All: "manufacturing scheduling" AND "deep reinforcement learning"} & 2014--2024, proceedings & 36 \\
    arXiv & \texttt{cat:cs.AI AND ("manufacturing scheduling" OR "job shop") AND "reinforcement learning"} & 2019--2024 & 41 \\
    \bottomrule
  \end{tabular}
\end{table}

\chapter{Data Extraction Templates}
\label{app:templates}
Structured artifacts live under \texttt{data/processed/} and are generated/validated through Python scripts. Table~\ref{tab:catalog-schema} documents the current schema for \texttt{study\_catalog.csv}; the accompanying \texttt{study\_summary.json} aggregates counts by year, domain, RL method, and KPI.

\begin{table}[t]
  \centering
  \caption{Schema for \texttt{data/processed/study\_catalog.csv}.}
  \label{tab:catalog-schema}
  \begin{tabular}{p{3cm} p{10cm}}
    \toprule
    Column & Description \\
    \midrule
    \texttt{paper\_id} & Stable identifier (prefix domain year). \\
    \texttt{title} & Official publication title. \\
    \texttt{year} & Publication year (YYYY). \\
    \texttt{manufacturing\_domain} & Categorized domain (job shop, flexible job shop, etc.). \\
    \texttt{rl\_method} & Primary RL approach (e.g., PPO actor-critic, cooperative MARL). \\
    \texttt{baseline} & Baseline heuristics/optimizers used for comparison. \\
    \texttt{kpis} & Comma-separated performance metrics (makespan, energy). \\
    \texttt{notes} & Evidence strength, dataset availability, or experimental remarks. \\
    \texttt{statistical\_testing} & Tests or intervals reported (if any). \\
    \texttt{deployment\_status} & Simulation, digital twin, pilot line, etc. \\
    \texttt{code\_available} & Whether public code is available. \\
    \texttt{simulator\_available} & Simulator/digital-twin availability. \\
    \bottomrule
  \end{tabular}
\end{table}

Researchers updating the dataset should:
\begin{enumerate}
  \item Run \texttt{python automation/summarize\_studies.py} to refresh aggregate statistics.
  \item Rebuild tables/charts via \texttt{make data} (runs summary, table rendering, KPI/deployment/year plots, and PRISMA figure).
  \item Document any new exclusion reasons directly in \texttt{data/prisma/screening\_log.csv}; every rejected full-text entry should include a brief justification.
\end{enumerate}

\chapter{Study Catalog Overview}
\label{app:overview}
Table~\ref{tab:overview} condenses the 55 curated studies into high-level counts. Domains align with the controlled vocabulary used throughout the thesis, and RL methods refer to the primary algorithm class. This appendix helps readers scan the corpus without opening the CSV files.

\begin{table}[t]
  \centering
  \caption{Snapshot of catalog composition (counts as of current PRISMA run).}
  \label{tab:overview}
  \begin{tabular}{p{4cm} p{3cm} p{6cm}}
    \toprule
    Category & Count & Notes \\
    \midrule
    Flexible job shops & 8 & Includes aerospace cells, curriculum RL, continual learning. \\
    Flow/hybrid shops & 6 & Covers multi-stage MARL, NSGA-II hybrids, battery lines. \\
    Semiconductor fabs & 12 & Encompasses EUV, supply-chain coordination, energy-aware dispatching. \\
    Energy/microgrid plants & 11 & Combines factory scheduling with carbon-aware microgrids. \\
    Regulated industries & 7 & Pharma, biopharma, remanufacturing, circular manufacturing. \\
    Robot/assembly cells & 5 & Human-aware actor-critic dispatching. \\
    Other specialized domains & 6 & Multi-plant energy, digital-twin orchestration, circular networks. \\
    \midrule
    Value-based RL (DQN variants) & 9 & Primarily early semiconductor/energy works. \\
    Policy gradient / PPO & 18 & Dominant in flexible job shops and energy-aware scheduling. \\
    SAC / DDPG & 10 & Common for microgrids and continuous controls. \\
    Multi-agent RL & 12 & Cluster tools, microgrids, robot cells. \\
    Hybrid RL+OR & 6 & CP-SAT refinement, NSGA-II warm starts. \\
    \bottomrule
  \end{tabular}
\end{table}

Counts will change as the corpus grows; regenerated tables ensure this appendix stays synchronized with the processed dataset.

\bigskip
\noindent\textbf{Per-study catalog.} Table~\ref{tab:catalog-list} provides a long-form view of every study, listing its domain, RL method, and reported KPIs. The table is generated automatically from \texttt{\detokenize{data/processed/study_catalog.csv}} to guarantee consistency.

\begin{center}
\small
\renewcommand{\arraystretch}{1.1}
\begin{minipage}{\linewidth}
\captionof{table}{Full study catalog (abridged metadata).}
\label{tab:catalog-list}
\begin{longtable}{p{1.8cm} p{2cm} p{2.2cm} p{2.2cm} p{2.2cm} p{1.8cm} p{3.5cm}}
\toprule
ID & Domain & RL method & Baselines & KPIs & Deployment & Notes \\ \midrule
\endfirsthead
\toprule
ID & Domain & RL method & Baselines & KPIs & Deployment & Notes \\ \midrule
\endhead
RL-JSSP-2020 & Flexible job shop & CNN-enhanced deep RL & SPT, ATC, tabu search & makespan, weighted tardiness & Simulation only & Taillard + real-world case; simulation only. \\ 
RL-GNN-2021 & Job shop & Graph attention actor-critic & SPT, EDD, priority dispatch & makespan, generalization gap & Simulation only & Demonstrates zero-shot transfer when job count varies. \\ 
RL-MARL-2021 & Flexible flow shop & Cooperative MARL policy gradients & Dispatching rules (FIFO, ATC) & average flow time, throughput & Simulation only & Evaluated under varying buffer capacities. \\ 
RL-PPO-2022 & Flexible job shop & PPO actor-critic & ATC, GA, MILP (small instances) & tardiness, rush order response time & Simulation only & Incorporates resource calendars and changeover costs. \\ 
RL-HYBRID-2023 & Flexible job shop & RL policy + CP-SAT refinement & CP-SAT only & makespan, feasibility rate & Simulation only & RL proposes machine assignments; OR ensures constraint satisfaction. \\ 
RL-ENERGY-2021 & Flow shop energy-aware & Dueling DQN & rule-based energy shedding & energy consumption, makespan & Simulation only & Considers time-of-use tariffs; maintains throughput within 2\%. \\ 
RL-MAINT-2022 & Assembly/flow line & Multi-objective RL (policy gradient) & deterministic maintenance scheduler & downtime, throughput, mean time between failures & Digital twin only & Adds machine health signals to state; tested on digital twin. \\ 
RL-SEMICON-2020 & Semiconductor fab & Double DQN & dispatching heuristics, MILP (small cases) & cycle time, throughput, WIP & Simulation only & Evaluated on 300mm fab model with re-entrant flow. \\ 
RL-HFLOW-2021 & Hybrid flow shop & Hierarchical RL (options+policies) & SPT, NEH, genetic algorithm & makespan, tardiness & Simulation only & Stage-level manager allocates buffers; sub-policies choose machines. \\ 
RL-MARL-ROBUST-2022 & Job shop (disturbed) & Decentralized MARL with resilience bonuses & ATC, robust tabu search & makespan, recovery time & Simulation only & Includes machine breakdown scenarios and re-training triggers. \\ 
RL-ENERGY-2023B & Flexible job shop & PPO-LSTM & EDD, rule-based energy throttling & energy consumption, tardiness & Simulation only & LSTM captures demand response signals for hourly tariffs. \\ 
RL-SEMICON-MARL-2022 & Semiconductor fab & Multi-agent actor-critic & dispatching rules, petri-net scheduler & throughput, transport utilization & Simulation only & Cluster-tool agents coordinate wafer transfers. \\ 
RL-BATCH-2021 & Batch process & Actor-critic with batching constraints & MILP baseline, heuristic batching & batch makespan, service level & Simulation only & Reward penalizes partial batches and changeovers. \\ 
RL-FLEXSIM-2024 & F flexible line (battery) & Graph RL (message passing) & simulation-optimized dispatching & throughput, buffer stability & Pilot line (HIL) & Policy trained in FlexSim digital twin and validated on pilot line. \\ 
RL-MULTIOBJ-2023 & Flow shop & Actor-critic + NSGA-II & NSGA-II alone, weighted-sum heuristics & makespan, energy, tardiness & Simulation only & Actor provides warm starts for NSGA-II to refine Pareto front. \\ 
RL-SEMICON-2023 & Semiconductor fab (EUV) & Policy gradient with feature shaping & dispatching heuristics, MILP & cycle time, energy & Simulation only & Addresses EUV scanner bottlenecks with policy gradient. \\ 
RL-MARL-ENERGY-2022 & Flexible flow shop (microgrid) & Multi-agent SAC & heuristic microgrid controller & energy consumption, throughput & Simulation only & Agents coordinate production and microgrid storage. \\ 
RL-FLEX-2024B & Flexible job shop & Continual learning PPO + human feedback & ATC, human experts & makespan, operator satisfaction & Pilot deployment (shadow mode) & Human preference data guides policy updates. \\ 
RL-SEMICON-TRANSFER-2023 & Semiconductor fab & Transfer learning with attention & dispatch rules across fabs & cycle time, throughput & Simulation only & Pre-trained on Fab A, fine-tuned on Fab B. \\ 
RL-MANUF-2024 & Aerospace assembly & Hierarchical actor-critic & Rule-based scheduling & throughput, takt compliance & Simulation only & Large-scale assembly line with multiple zones. \\ 
RL-ROBOTCELL-2023 & Robot cell & Actor-critic with shared resources & rule-based robot scheduling & cycle time, collision risk & Simulation only & Focus on collaborative robot cells with shared fixtures. \\ 
RL-ENERGY-SEMICON-2022 & Semiconductor fab SAC & energy-aware heuristics & energy consumption, cycle time & Adds energy tariffs to wafer scheduling in fabs. & No & none reported \\ 
RL-BEV-2023 & Battery EV module line & Multi-objective actor-critic & weighted heuristics & makespan, energy, quality & Simulation only & Optimizes EV module assembly considering quality metrics. \\ 
RL-SEMICON-2024 & Semiconductor fab & Graph RL with attention & dispatch heuristics, MILP & cycle time, transport utilization & Simulation only & Graph policy coordinates multiple cluster tools. \\ 
RL-HYBRID-2024 & Hybrid flow shop & Hierarchical multi-objective RL & NSGA-II, weighted heuristics & makespan, energy, robustness & Simulation only & Combines stage-level manager with energy/robustness rewards. \\ 
RL-FLEX-2025 & Aerospace flexible cell & Transfer learning actor-critic & human schedulers, ATC & throughput, takt compliance & Pilot line (shadow mode) & Transfer across aerospace cells with human feedback. \\ 
RL-MICROGRID-2024 & Flexible flow shop (microgrid) & Multi-objective SAC & heuristic microgrid control & energy, carbon, throughput & Simulation only & Adds carbon pricing to microgrid scheduling. \\ 
RL-SEMICON-2025 & Semiconductor fab (EUV) & Self-play PPO & dispatch heuristics & cycle time, energy & Simulation only & Self-play between virtual schedulers. \\ 
RL-BATCH-2024 & Continuous batch & Policy gradient with constraints & MILP, heuristics & batch makespan, quality & Simulation only & Continuous pharma process with RL control. \\ 
RL-SEMICON-MULTIOBJ-2024 & Semiconductor fab & Multi-objective actor-critic & heuristics, weighted sums & cycle time, energy & Simulation only & Energy + cycle-time optimization in fabs. \\ 
RL-FLEXSIM-2025 & F flexible line (battery) & RL + streaming digital twin & rule-based scheduling & throughput, buffer stability & Pilot line (HIL) & Policy adapts to streaming twin feedback. \\ 
RL-SEMICON-CHAIN-2025 & Semiconductor supply chain & Hierarchical RL & MRP, heuristics & cycle time, WIP, delivery & Simulation only & Links wafer, test, assembly scheduling. \\ 
RL-FLEX-DUAL-2024 & Flexible job shop & Dual-agent actor-critic & ATC, heuristics & makespan, tardiness & Simulation only & Agents collaborate: job selector + machine assigner. \\ 
RL-MICROGRID-2025 & Microgrid manufacturing & Multi-agent SAC with carbon pricing & microgrid heuristics & energy, carbon, throughput & Simulation only & Extends microgrid RL with carbon credits. \\ 
RL-SEMICON-LOT-2024 & Semiconductor fab (advanced nodes) & Policy gradient with lot-sizing & MILP, heuristics & cycle time, lot smoothing & Simulation only & Lot-sizing decisions for advanced nodes. \\ 
RL-BATCH-CONT-2025 & Hybrid pharma line & Actor-critic with mode-switching & MILP, heuristics & quality, throughput & Simulation only & Handles hybrid batch/continuous modes. \\ 
RL-ENERGY-MANUF-2024 & Multi-plant manufacturing & Hierarchical PPO & energy heuristics & energy, carbon, throughput & Simulation only & Coordinates multi-plant energy constraints. \\ 
RL-ROBOTCELL-2024 & Robot cell & Multi-agent actor-critic & rule-based robot scheduler & cycle time, collision risk & Simulation only & Expanded robot cell scenario with shared fixtures. \\ 
RL-SEMICON-COOP-2025 & Semiconductor fab & Cooperative MARL & dispatch heuristics & cycle time, tardiness & Simulation only & Agents coordinate wafer dispatch decisions. \\ 
RL-FJSSP-2025 & Flexible job shop & Graph curriculum RL & ATC, GA & makespan, tardiness & Simulation only & Curriculum scaling to 200+ operations; zero-shot tests. \\ 
RL-SEMICON-YIELD-2024 & Semiconductor fab & Actor-critic with yield penalty & heuristics, MILP & cycle time, yield & Simulation only & Integrates yield targets into wafer scheduling. \\ 
RL-MICROGRID-2025B & Multi-microgrid manufacturing & Hierarchical multi-agent RL & microgrid heuristics & energy, carbon, throughput & Simulation only & Coordinates multiple microgrids via hierarchical RL. \\ 
RL-BIOPHARMA-2024 & Biopharma batch & Distributional RL & MILP, heuristics & makespan, service level, yield & Simulation only & Handles stochastic yields in biopharma batching. \\ 
RL-SEMICON-SUPPLY-2025 & Semiconductor supply chain & Multi-objective actor-critic & MRP, heuristics & cycle time, delivery, energy & Simulation only & Coordinates wafer + assembly scheduling across fabs. \\ 
RL-ROBOTCELL-2025 & Robot cell & Human-aware actor-critic & rule-based HR scheduling & cycle time, operator load & Simulation only & Human-aware scheduling in collaborative cells. \\ 
RL-MANUF-2025 & Remanufacturing line & Actor-critic with reman options & heuristics, MILP & throughput, quality & Simulation only & Handles remanufacturing disassembly + assembly. \\ 
RL-ENERGY-2025 & Multi-carrier energy manufacturing & Actor-critic with hydrogen/electricity & energy heuristics & energy cost, carbon & Simulation only & Optimizes electricity + hydrogen carriers. \\ 
RL-SEMICON-2030 & Semiconductor fab (high NA) & Policy gradient + self-play & EUV heuristics & cycle time, energy & Simulation only & Targets next-gen high-NA EUV tools. \\ 
RL-FJSSP-2030 & Flexible job shop & Meta-RL with adaptation & ATC, heuristics & makespan, tardiness & Simulation only & Meta-RL adapts quickly to new instances. \\ 
RL-MICROGRID-2030 & Multi-factory microgrid & Hierarchical multi-agent RL & microgrid heuristics & energy, carbon, throughput & Simulation only & Coordinates storage across multiple factories. \\ 
RL-PHARM-2026 & Personalized pharma line & Actor-critic with patient-level constraints & MILP, heuristics & service level, quality & Simulation only & Handles personalized production scheduling. \\ 
RL-SEMICON-COBOT-2026 & Semiconductor fab (cobots) & Human-aware MARL & dispatch heuristics & cycle time, safety & Simulation only & Cobots integrated into fab scheduling. \\ 
RL-ROBOTCELL-2030 & Robot cell & Multi-line actor-critic & rule-based robot scheduling & throughput, collisions & Simulation only & Coordinates robot cells across lines. \\ 
RL-DIGITALTWIN-2026 & Smart factory & RL + streaming twins & rule-based scheduler & throughput, buffer stability & Pilot line (HIL) & Orchestrates multiple streaming digital twins. \\ 
RL-MANUF-2030 & Circular manufacturing & Multi-objective RL & heuristics, MIP & throughput, recycling rate & Simulation only & Schedules production + remanufacturing in circular networks. \\ 
\bottomrule
\end{longtable}

\end{minipage}
\end{center}

\chapter{Interview Protocol}
\label{app:interviews}
Semi-structured interviews supported the qualitative assessments in Chapters~\ref{ch:casestudies} and~\ref{ch:adoption}. Each conversation covered four themes: (i) current scheduling workflows and pain points, (ii) digital infrastructure readiness, (iii) governance and compliance requirements, and (iv) success criteria for RL pilots. Interviewees included planners from battery manufacturing, semiconductor fabs, aerospace assembly, and pharmaceutical operations. Notes were anonymized and synthesized into theme clusters (reward design, interpretability, digital-twin fidelity), which informed the cross-domain lessons.

\chapter{Glossary of Terms}
\label{app:glossary}
\begin{description}
  \item[ATC] Apparent Tardiness Cost, a heuristic dispatching rule balancing due dates and processing times.
  \item[Digital Twin] A living simulation of physical assets that exchanges state data with the real factory, often implemented in FlexSim, AnyLogic, or proprietary fab models.
  \item[HIL] Hardware-in-the-loop testing, where RL policies issue commands to real controllers while the legacy scheduler remains in charge.
  \item[MARL] Multi-agent reinforcement learning, wherein multiple policies coordinate via shared rewards or communication protocols.
  \item[Reward Council] Cross-functional forum proposed in Chapter~\ref{ch:casestudies} where stakeholders tune reward weights before retraining policies.
\end{description}

\chapter{Computation Environment}
\label{app:environment}
All automation scripts run on Python~3.10 with pandas, matplotlib, and seaborn. LaTeX compilation uses TeX Live 2019 with \texttt{latexmk}. Slidev builds require Node.js 18+ and npm. Experiments referenced in Chapter~\ref{ch:toolchain} used Ubuntu 22.04 workstations equipped with 32\,GB RAM; GPU acceleration was optional because most policy training occurred on remote clusters described within individual studies. Reproducing the thesis from scratch involves cloning the repository, installing Python dependencies listed in \texttt{automation/README.md}, running \texttt{make data}, and finally invoking \texttt{make thesis} and \texttt{make slidev}.

\chapter{Key Performance Indicator Glossary}
\label{app:kpi}
\begin{description}
  \item[Makespan] Completion time of the final job; often reported as average across test instances or normalized against benchmarks.
  \item[Total tardiness] Sum of positive lateness values; weighted variants emphasize high-priority orders.
  \item[Throughput] Jobs completed per horizon; flow shops frequently report percentage gains relative to NEH or GA baselines.
  \item[Energy/Demand charge] Kilowatt-hours consumed or cost-based proxies (USD, EUR); microgrid studies also track carbon intensity.
  \item[Resilience metrics] Recovery time after disruption, number of schedule adjustments, or variance of KPI under stochastic perturbations.
\end{description}
When comparing studies, ensure KPIs share units; Chapter~\ref{ch:metaanalysis} consolidates reported values where feasible.

\chapter{Hyperparameter Reference}
\label{app:hyper}
Although each paper tunes hyperparameters differently, several patterns recur:
\begin{itemize}
  \item PPO clip ratios between 0.1 and 0.3, entropy coefficients 0.01--0.05, and value-loss coefficients near 0.5 for job shops and energy-aware contexts.
  \item SAC temperature parameters auto-tuned with target entropy equal to $-|\mathcal{A}|$, enabling stable training in microgrids \cite{RL-MARL-ENERGY-2022}.
  \item Replay buffers of at least 500k transitions for flow shops to capture diverse queue states; prioritized replay helps when reward signals are sparse.
  \item Curriculum schedules that increase job counts every 1--2 million steps, preventing catastrophic forgetting when scaling to industrial instance sizes.
\end{itemize}
Documenting these values helps practitioners benchmark compute requirements and reproduce published results.

%%% Local Variables: 
%%% mode: latex
%%% TeX-master: "thesis.tex"
%%% End: 
