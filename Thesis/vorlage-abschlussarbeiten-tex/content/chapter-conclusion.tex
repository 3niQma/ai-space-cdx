\chapter{Conclusion}
\label{ch:conclusion}

This thesis delivered a PRISMA-guided synthesis of RL techniques for manufacturing scheduling, emphasized exclusively on factory contexts such as job shops, flow shops, flexible systems, and semiconductor fabs. Key insights include the rapid adoption of deep RL architectures, the rise of hybrid RL+OR workflows, and the nascent but important focus on sustainability-aware scheduling. Equally significant is the demonstration of a multi-agent LLM pipeline that documents every review stage, ensuring transparency and reproducibility.

The curated corpus shows how different domains demand tailored modeling choices: Taillard-inspired flexible job shops benefit from graph encoders and continual learning \cite{RL-GNN-2021,RL-FLEX-2024B}, semiconductor fabs require hybrid dispatchers that respect batching, energy, and yield constraints \cite{RL-SEMICON-2024,RL-SEMICON-MULTIOBJ-2024}, and microgrid-integrated factories lean on multi-agent SAC policies to balance throughput with carbon budgets \cite{RL-MARL-ENERGY-2022,RL-MICROGRID-2025}. Specialized sectors—robot cells, pharma, circular manufacturing—demonstrate that RL schedulers can honor safety and regulatory requirements when reward shaping is explicit \cite{RL-ROBOTCELL-2025,RL-BATCH-CONT-2025,RL-MANUF-2030}. Capturing these nuances within a single dataset provides a baseline for future quantitative meta-analyses once the study count reaches the planned 200 inclusions.

The accompanying automation scripts, datasets, and Slidev presentation provide a reusable toolkit for research teams who need to keep stakeholders informed as the literature evolves. Future iterations will populate the remaining data placeholders with the targeted 200 studies and will extend the agent framework with automated visualization and citation verification components. Beyond content expansion, the roadmap includes: (i) harmonizing benchmark definitions so KPIs can be normalized across proprietary twins; (ii) enriching the Slidev deck with auto-generated charts from \texttt{study\_summary.json}; and (iii) integrating reproducibility badges that highlight which studies share code, simulators, or deployment evidence. By converging systematic-review rigor with auditable automation, the project illustrates how AI agents can accelerate literature synthesis without sacrificing methodological discipline.

\section*{Recommendations for Practitioners}
Manufacturing leaders considering RL deployments should start by cataloging available simulators and KPIs, mirroring the structure of \texttt{\detokenize{data/processed/study_catalog.csv}}. Pilot projects benefit from hybrid RL+OR architectures that enforce feasibility, complemented by interpretability layers (rule extraction, saliency visualization) to build trust with planners. Energy-aware factories should incorporate tariff forecasts and carbon budgets into rewards from day one—retrofits are more difficult once policies enter production. Finally, invest early in reproducible tooling: the combination of \texttt{make data}, \texttt{make thesis}, and \texttt{make slidev} provides a blueprint for keeping analyses, documentation, and stakeholder communications synchronized as new studies emerge.
